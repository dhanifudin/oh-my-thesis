\noindent
\changeSize{\JudulInggris}

\noindent
\begin{table}
    \centering
    \begin{tabular}{l l l}
        Name&: & \penulis \\
        Student Identity Number&: & \nrp \\
        Supervisor&: & \pembimbingSatu \\
        %Co-supervisor&: & \pembimbingDua \\
    \end{tabular}
\end{table}

\noindent
\changeSize{ABSTRACT}

The development of information and communication technology affect the way
people to interact with objects that associated. One of the processess of
interaction needed in mobile environment is the process of tracking.  Tracking
is the process of observing people or moving objects continuously in which the
objects were observed continuously monitored both positions and activities. The
ideal tracking process can send the location changes constantly in the
changeable conditions. However, such tracking systems are generally less
efficient because it can spend resource of power and bandwidth requiring a more
efficient process. Traditional tracking system less efficient to be developed
into multi targets tracking infrastructure on mobile devices where either
tracker or tracked object is more than one. It takes a loosely coupled
communication mechanism.

In this research, researcher developed a multi target tracking mechanism based
on publish-subscribe architecture to save bandwidth and resource usage.
Improvement mechanism was made with content of interest. Data communication is
only done for the neccessary content. The result of research show the adaptive
tracking multiple target has performance efficiency. This is shown with the
bandwidth savings of 44 - 60 \%, CPU savings 57 - 71 \% and latency efficiency
14 - 73 \%.

\noindent \\ \bo{Keywords}:
Multi Target Tracking, Publish-Subscribe, efficiency.

\cleardoublepage
