\noindent
\changeSize{\JudulInggris}

\noindent 
\begin{table}
    \centering
    \begin{tabular}{l l l}
        Name&: & \penulis \\
        Student Identity Number&: & \nrp \\
        Supervisor&: & \pembimbingSatu \\
        %Co-supervisor&: & \pembimbingDua \\
    \end{tabular}
\end{table}

\noindent
\changeSize{ABSTRACT}

The development of information and communication technology affect the way
people to interact with objects that associated. One of the processess of
interaction needed in mobile environment is the process of tracking.
Tracking is the process of observing people or moving objects continuously in
which the objects were observed continuously monitored both positions and
activities. The ideal tracking process can send the location changes constantly
in the changeable conditions. However, such tracking systems are generally less
efficient because it can spend resource of power and bandwidth requiring a
more efficient process.

Traditional tracking system less efficient to be developed into multi targets
tracking infrastructure on mobile devices where either tracker or tracked
object is more than one.  It takes a loosely coupled communication mechanism.
Publish-Subscribe have the advantages of decoupling in time, space and
synchronization. This interaction makes the publish-subscribe is ideal on a
dynamic large scale communication. Other efficiencies can conducted by doing
the tracking adaptively with context awareness.  Location updates based on
conditions of tracking that given by tracker and tracked object conditions
adaptively.  

\noindent \\ \bo{Keywords}:
Multi Target Tracking, Publish-Subscribe, Context Awareness

\cleardoublepage
