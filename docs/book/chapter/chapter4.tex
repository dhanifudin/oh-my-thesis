\chapter{\babEmpat}

Bab ini memaparkan bagaimana mengimplementasikan metode yang diusulkan pada bab
sebelumnya ke dalam bentuk kode program, mengimplementasikan ide dari penelitian
ini yang dievaluasi dalam \testbed~serta hasil yang didapat dievaluasi dan
dianalisa dari percobaan yang telah dilakukan berdasarkan beberapa skenario.

% Uji Coba {{{ %
\section{Uji Coba}

% Lingkungan Uji Coba {{{ %
\subsection{Lingkungan Uji Coba}

Metode yang diusulkan dievaluasi dalam lingkungan \testbed~yang dibangun
sendiri. \Testbed~yang dibangun merupakan komunikasi \server~dan \client.
\Client~dibagi menjadi dua macam, yaitu: \publiser~\client~dan
\subscriber~\client. \Testbed~dibangun dalam lingkungan pemrograman \nodejs.

Sedangkan spesifikasi perangkat keras dideskripsikan sebagai berikut.
Spesifikasi pada server, \processor~Intel(R) Xeon(R) CPU E5-2630L 2.4GHz, RAM
(\ram) 512Mb, Swap 1GB dengan sistem operasi CentOS 6.6 yang divirtualisasikan
oleh KVM pada Digital Ocean. Sedangkan untuk spesifikasi pada client,
\processor~Intel Core i5-3427U CPU 1.8Ghz, RAM (\ram) 4Gb, dengan sistem operasi
OS X 10.10.4 (Yosemite), Macbook-Air model tahun 2011.

% }}} Lingkungan Uji Coba %

% Skenario Pengujian {{{ %
\subsection{Skenario Pengujian}

Skenario pengujian pada penelitian ini dibagi menjadi tiga bagian. Skenario
pengujian ini dilakukan untuk mengetahui kinerja \tracking~dengan model
interaksi \pubsub.

\begin{enumerate}
  [label=\alph*.
  ,noitemsep
  ,nolistsep
  ,leftmargin=0cm
  ,itemindent=.5cm
  ,listparindent=\parindent
  ]

  \item Uji coba \bandwidth

  \item Uji coba \latency

  \item Uji coba sumber daya perangkat bergerak

\end{enumerate}

% }}} Skenario Pengujian %

% Parameter Pengujian {{{ %
\subsection{Parameter Pengujian}

Sesuai dengan tujuan penelitian ini, yakni untuk membangun sistem
\tracking~multi target serta \f{update protocol} yang bersifat adaptif berbasis
\pubsub~dalam lingkungan bergerak. Untuk mengetahui kualitas layanan, perlu
adanya parameter yang dapat mengidentifikasi perbaikan kualitas layanan model
interaksi \pubsub~dalam lingkungan bergerak. Parameter-parameter yang digunakan
dalam penelitian ini adalah \bandwidth, \latency~dan sumber daya baterai.

\begin{enumerate}
  [label=\alph*.
  ,noitemsep
  ,nolistsep
  ,leftmargin=0cm
  ,itemindent=.5cm
  ,listparindent=\parindent
  ]

  \item Penggunaan \bandwidth

  Uji coba parameter penggunaan \bandwidth~dilakukan dengan mengukur penggunaan
  \bandwidth~ketika terjadi interaksi antara \server~dan \client. Uji coba akan
  dilakukan sebanyak dua kali yakni menggunakan \f{update protocol} yang
  non-adaptif dan adaptif dengan model interaksi \pubsub. Hasil dari
  masing-masing pengujian akan dibandingan sehingga akan terlihat perbedaan
  penggunaan \bandwidth~antara kedua uji coba.

  \item \Latency~(Waktu respon)

  Uji coba parameter waktu respon dilakukan dengan menghitung waktu antara
  \publish~lokasi dikirimkan dari \publisher~ke \server~sampai diterima oleh
  \subscriber~berdasarkan dasar ketertarikan. Uji coba akan dilakukan sebanyak
  dua kali yakni menggunakan \f{update protocol} yang non-adaptif dan adaptif
  dengan model interaksi \pubsub. Hasil dari masing-masing tiap pengujian akan
  dibandingkan untuk kemudian dapat dianalisa.

  \item Sumber daya baterai
    \[TODO\]

\end{enumerate}

% }}} Parameter Pengujian %

% }}} Uji Coba %

% Hasil Uji Coba dan Analisis {{{ %
\section{Hasil Uji Coba dan Analisis}
% }}} Hasil Uji Coba dan Analisis %
